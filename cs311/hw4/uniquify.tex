\documentclass[12pt,letterpaper]{article}
\usepackage[margin=1in]{geometry}
\usepackage{fancyhdr}
\usepackage[utf8]{inputenc}
\usepackage{palatino}
\usepackage{microtype}
\usepackage{hyperref}
\usepackage{graphicx}
\usepackage{lastpage}
\usepackage[hang,bf,small]{caption}
\usepackage{titlesec}
\usepackage{amsmath,amssymb,amsthm}

\renewcommand{\headrulewidth}{0pt}
\fancyfoot{}
\fancyfoot[C]{\thepage}
\pagestyle{fancy}

% \titleformat{\section}{\bfseries\MakeUppercase}{\arabic{\thesection}}{1em}{}
% \titleformat{\subsection}{\bfseries}{\arabic{\thesection}.\arabic{\thesubsection}}{1em}{}
% \titleformat{\subsubsection}{\itshape}{\arabic{\thesection}.\arabic{\thesubsection}.\arabic{\thesubsubsection}}{1em}{}

\setlength{\parindent}{0cm}
\setlength{\parskip}{1em}

% \captionsetup[figure]{labelfont=it,font=it}
% \captionsetup[table]{labelfont={it,sc},font={it,sc}}

\hypersetup{colorlinks,
    linkcolor = black,
    citecolor = black,
    urlcolor  = black}
\urlstyle{same}


\begin{document}

Soo-Hyun Yoo \\
CS311 \\
Homework 4 \\
5 November 2012


\section*{Design}

This project implements parallel mergesort using multiple sort processes.

The number of parallel processes to spawn is passed as a command line argument. These sort processes are forked off of the main process and connected via pipes.

Once the forks are complete, the parent process parses a list of words (delimited by newlines) from stdin and hands out each word to the sort processes in a round-robin style. Once it is done reading from stdin, it closes the pipes, which signals the sort processes to sort.

The outputs of the sort processes are piped back to the parent process, which counts the frequency of each word and prints them to stdout on the fly.


\section*{Worklog}

\subsection*{Nov. 3}

\begin{itemize}
	\item Add initial files for HW 4.
	\item Add TeX file.
	\item Add simple random string generator.
	\item Write up rough outline of design.
	\item argv is char** not int**.
\end{itemize}

\subsection*{Nov. 4}

\begin{itemize}
	\item Outline code in main.
	\item Add header files for functions.
	\item Include headers.
	\item Read from stdin.
	\item Update design.
	\item Fork children.
	\item Add pipes.
\end{itemize}

\subsection*{Nov. 5}

\begin{itemize}
	\item Minor edits.
	\item Don't use c99 standard.
	\item Pipes finally work.
	\item Sort returns correctly.
	\item Parser works.
	\item Keep parser simple.
	\item Implement merger.
	\item Merger should not print duplicates.
	\item Print results at the end.
	\item Remove obsolete comments and add new ones.
	\item Add comments.
	\item Don't parse newlines.
	\item Merge counts word frequency.
	\item Update randgen.py to take command-line argument.
	\item Make merge print the last line.
	\item Generate longer words.
	\item Remove obsolete comments and old code.
	\item Add lists target.
	\item Implement simple usage() function.
	\item Generate duplicate word list.
	\item Add README.
	\item Finish writeup.
\end{itemize}


\section*{Challenges}

It took me a while to figure out that I need to fflush the stream after each fputs. This was hard to debug when I had not yet realized the problem and concurrently had pipes from the parent to the child and vice versa. Starting from a basic parent with a single child and only one pipe (and carefully building up the complexity) helped resolve the problem.


\section*{Answers to questions}

\begin{itemize}
	\item This assignment familiarized me with dealing with pipes in C.
	\item I tested my solution by feeding it lists of words large and small, unique and non-unique. These lists are generated by randgen.py.
	\item I learned how to set up and use pipes. It was also interesting to learn that two similar-sounding functions, fgets and fputs, have different failure return values (NULL and EOF, respectively). Finally, I learned that in order to use gdb to debug a program A that takes its input via a pipe from another process B, I can:

		\begin{enumerate}
			\item Use mkfifo to create a FIFO.
			\item Pipe the output of B into the FIFO.
			\item Run A in gdb.
			\item In the gdb prompt, run A and pipe in from the FIFO.
		\end{enumerate}
\end{itemize}

\end{document}
