% template created by: Russell Haering. arr. Joseph Crop
\documentclass[12pt,letterpaper]{article}
\usepackage{anysize}
\marginsize{2cm}{2cm}{1cm}{1cm}
\usepackage{listings}
\usepackage{color}

\definecolor{dkgreen}{rgb}{0,0.6,0}
\definecolor{gray}{rgb}{0.5,0.5,0.5}
\definecolor{mauve}{rgb}{0.58,0,0.82}

\lstset{
	language={[x86masm]Assembler},
	basicstyle=\footnotesize,           % the size of the fonts that are used for the code
	numbers=left,                   % where to put the line-numbers
	numberstyle=\tiny\color{gray},  % the style that is used for the line-numbers
	stepnumber=1,                   % the step between two line-numbers. If it's 1, each line~
	                                % will be numbered
	numbersep=5pt,                  % how far the line-numbers are from the code
	backgroundcolor=\color{white},      % choose the background color. You must add \usepackage{color}
	showspaces=false,               % show spaces adding particular underscores
	showstringspaces=false,         % underline spaces within strings
	showtabs=false,                 % show tabs within strings adding particular underscores
	frame=single,                   % adds a frame around the code
	rulecolor=\color{black},        % if not set, the frame-color may be changed on line-breaks within not-black text (e.g. commens (green here))
	tabsize=2,                      % sets default tabsize to 2 spaces
	captionpos=b,                   % sets the caption-position to bottom
	breaklines=true,                % sets automatic line breaking
	breakatwhitespace=false,        % sets if automatic breaks should only happen at whitespace  
	title=\lstname,                   % show the filename of files included with \lstinputlisting;
	                                % also try caption instead of title
	keywordstyle=\color{blue},          % keyword style
	commentstyle=\color{dkgreen},       % comment style
	stringstyle=\color{mauve},         % string literal style
	escapeinside={\%*}{*)},            % if you want to add LaTeX within your code
	morekeywords={*,...}               % if you want to add more keywords to the set
}

\begin{document}

\begin{titlepage}
    \vspace*{4cm}
    \begin{flushright}
    {\huge
        ECE 375 Lab 4\\[1cm]
    }
    {\large
        Simple Interrupts
    }
    \end{flushright}
    \begin{flushleft}
    Lab Time: Wednesday 5-7
    \end{flushleft}
    \begin{flushright}
    Soo-Hyun Yoo

    \vfill
    \rule{5in}{.5mm}\\
    TA Signature
    \end{flushright}

\end{titlepage}

\section*{Study Questions}

\begin{enumerate}
	\item Monitoring external inputs using interrupts uses clock cycles more
		efficiently and therefore is a good option for simple devices designed
		for long-term operation on limited power. However, since interrupts
		require context switching to service, they may not be a good option for
		programs with many external events to process.

		C is often preferable for larger projects where the abstraction it
		provides helps the programmer more easily see the bigger picture. It is
		also faster to produce effective, understandable code in C, albeit
		possibly at the cost of efficiency compared to assembly.

	\item The wait function could use a timer that triggers a software
		interrupt, which would bring the robot out of wait. If this interrupt
		is placed lower on the interrupt vector list, it can be called from
		within one of the external interrupts.
\end{enumerate}

\end{document}

