% template created by: Russell Haering. arr. Joseph Crop
\documentclass[12pt,letterpaper]{article}
\usepackage{anysize}
\marginsize{2cm}{2cm}{1cm}{1cm}

\begin{document}

\begin{titlepage}
    \vspace*{4cm}
    \begin{flushright}
    {\huge
        ECE 375 Lab 1\\[1cm]
    }
    {\large
        Introduction to AVR Tools
    }
    \end{flushright}
    \begin{flushleft}
    Lab Time: Wednesday 5-7
    \end{flushleft}
    \begin{flushright}
    Soo-Hyun Yoo

    \vfill
    \rule{5in}{.5mm}\\
    TA Signature
    \end{flushright}

\end{titlepage}

\begin{enumerate}
	\item The template uses the \LaTeX default font, which is 12pt Computer
		Modern. The source code is monospaced.
	\item Writing code in a language that can be cross-compiled makes it
		possible to run the same algorithms on different hardware platforms.
		For example, the C code written for the AVR could be ported to operate
		on the ARM architecture, on a PIC, and so on.

		One problem with this is that sometimes, libraries for different
		hardware architectures have different software components.
		Functionalities available on one architecture may be named differently
		or altogether unavailable on another. For example, floating point
		operations written for an ARM board may need to be rewritten or
		substituted for operation on an ATmega board with integer-only math.
\end{enumerate}

\end{document}
